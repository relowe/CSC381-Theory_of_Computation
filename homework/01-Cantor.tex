\documentclass{article}
\usepackage{fullpage}
\usepackage[T1]{fontenc}
\usepackage{titling}
\usepackage{graphicx}
\usepackage{enumitem}

\setlength{\droptitle}{-8em}   % This is your set screw
\title{Cantor Problem Set}
\date{Due: January 22, 2020}

\begin{document}
\maketitle

For each problem, construct a proof which either proves or disproves
the proposition.  Your answers must be typeset in latex, and if you use
outside sources, you must cite your work.  For full credit, you must
attempt all of these problems.  In addition to attempting all problems,
you will be graded on your best 3 easier problems and your best
2 harder problems.  (In other words, you must try them all, but you
must succeed on at least 5 of them.) Bring a hardcopy to class for
submission.

\section*{Easier Problems}
\begin{enumerate}
    \item Cantor's diagonal argument works with more than two symbols.
        For instance, instead of $m$ and $w$, we could use
        0,1,2,3,4,5,6,7,8,9 and still obtain the same result.
    \item $\sqrt{2}$ is a non-algebraic number.
    \item The set of all rational numbers are countable.
    \item The set of all algebraic numbers is countable.
    \item The set of all integers (positive and negative) is
        countable.
    \item There are necessarily more rational numbers than irrational
        numbers.
\end{enumerate}

\section*{Harder Problems}
\begin{enumerate}
    \item The set of all finite but unbounded sequences of $m$, $w$
        are countable.
    \item Every transcendental (non-algebraic) number must contain all
        digits of its number base at least one time.  For instance, in
        decimal, a transcendental number must have all digits from 0 to
        9 somewhere in its infinite sequence.
    \item The set of all prime numbers has the same cardinality as the
        set of all algebraic real numbers.
    \item Besides infinite recursion of partitions and the diagonal argument,
        find a proof that the cardinality of natural numbers is strictly
        less than the cardinality of real numbers.
    \item The set of infinities is itself uncountable.
\end{enumerate}
\end{document}
