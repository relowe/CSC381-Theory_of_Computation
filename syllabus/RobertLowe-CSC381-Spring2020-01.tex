% Don't touch this %%%%%%%%%%%%%%%%%%%%%%%%%%%%%%%%%%%%%%%%%%%
\documentclass[11pt]{article}
\usepackage{fullpage}
\usepackage[left=1in,top=1in,right=1in,bottom=1in,headheight=3ex,headsep=3ex]{geometry}
\usepackage{graphicx}
\usepackage{float}
\usepackage{longtable}

\newcommand{\blankline}{\quad\pagebreak[2]}
%%%%%%%%%%%%%%%%%%%%%%%%%%%%%%%%%%%%%%%%%%%%%%%%%%%%%%%%%%%%%%

% Modify Course title, instructor name, semester here %%%%%%%%
\title{CSC381: Introduction to Computer Science II}
\author{Dr. Robert Lowe}
\date{Spring, 2020}


% Don't touch this %%%%%%%%%%%%%%%%%%%%%%%%%%%%%%%%%%%%%%%%%%%
\usepackage[sc]{mathpazo}
\linespread{1.05} % Palatino needs more leading (space between lines)
\usepackage[T1]{fontenc}
\usepackage[mmddyyyy]{datetime}% http://ctan.org/pkg/datetime
\usepackage{advdate}% http://ctan.org/pkg/advdate
\newdateformat{syldate}{\twodigit{\THEMONTH}/\twodigit{\THEDAY}}
\newsavebox{\MONDAY}\savebox{\MONDAY}{Mon}% Mon
\newcommand{\week}[1]{%
%  \cleardate{mydate}% Clear date
% \newdate{mydate}{\the\day}{\the\month}{\the\year}% Store date
  \paragraph*{\kern-2ex\quad #1, \syldate{\today} - \AdvanceDate[4]\syldate{\today}:}% Set heading  \quad #1
%  \setbox1=\hbox{\shortdayofweekname{\getdateday{mydate}}{\getdatemonth{mydate}}{\getdateyear{mydate}}}%
  \ifdim\wd1=\wd\MONDAY
    \AdvanceDate[7]
  \else
    \AdvanceDate[7]
  \fi%
}
\usepackage{setspace}
\usepackage{multicol}
%\usepackage{indentfirst}
\usepackage{fancyhdr,lastpage}
\usepackage{url}
\pagestyle{fancy}
\usepackage{hyperref}
\usepackage{lastpage}
\usepackage{amsmath}
\usepackage{layout}

\lhead{}
\chead{}
%%%%%%%%%%%%%%%%%%%%%%%%%%%%%%%%%%%%%%%%%%%%%%%%%%%%%%%%%%%%%%

% Modify header here %%%%%%%%%%%%%%%%%%%%%%%%%%%%%%%%%%%%%%%%%
\rhead{\footnotesize CSC3810-01 Spring 2020}

%%%%%%%%%%%%%%%%%%%%%%%%%%%%%%%%%%%%%%%%%%%%%%%%%%%%%%%%%%%%%%
% Don't touch this %%%%%%%%%%%%%%%%%%%%%%%%%%%%%%%%%%%%%%%%%%%
\lfoot{}
\cfoot{\small \thepage/\pageref*{LastPage}}
\rfoot{}

\usepackage{array, xcolor}
\usepackage{color,hyperref}
\definecolor{clemsonorange}{HTML}{EA6A20}
\hypersetup{colorlinks,breaklinks,linkcolor=clemsonorange,urlcolor=clemsonorange,anchorcolor=clemsonorange,citecolor=black}

\begin{document}

\maketitle

\blankline

\begin{tabular*}{.93\textwidth}{@{\extracolsep{\fill}}lr}
%%%%%%%%%%%%%%%%%%%%%%%%%%%%%%%%%%%%%%%%%%%%%%%%%%%%%%%%%%%%%%

% Modify information %%%%%%%%%%%%%%%%%%%%%%%%%%%%%%%%%%%%%%%%%
E-mail: \texttt{robert.lowe@maryvillecollege.edu} & Office Phone: 865-981-8169 \\

 Office Hours: MWF 1:00PM -- 2:00PM, TR 3:00PM -- 4:00PM  &  Class
 Hours: MWF 11:00 -- 11:50\\
 Office: SSC 214 & Class Room: SSC 231\\
\hline
\end{tabular*}

\vspace{5 mm}

% First Section %%%%%%%%%%%%%%%%%%%%%%%%%%%%%%%%%%%%%%%%%%%%
\section*{Course Description}
A study of theoretical models of computing, including finite state
machines, pushdown automata, context-free grammars, and Turing
machines. The concepts of decidability, complexity theory, and
NP-Completeness will be studied in depth. 


% Second Section %%%%%%%%%%%%%%%%%%%%%%%%%%%%%%%%%%%%%%%%%%%
\section*{Required Materials}
\begin{itemize}
    \item Various papers, which will be distributed in class.
\end{itemize}

% Third Section %%%%%%%%%%%%%%%%%%%%%%%%%%%%%%%%%%%%%%%%%%%%
\section*{Prerequisites}
\begin{itemize}
    \item CSC2310 Discrete Structures
    \item Courage
\end{itemize}


% Fourth Section %%%%%%%%%%%%%%%%%%%%%%%%%%%%%%%%%%%%%%%%%%%
\section*{Course Goals}
The big-picture goal of this course is for you to become well
acquainted with abstract mathematical literature and learn how to
treat mathematics as a participatory subject. We will be reading the
papers which started our field, and along the way we will cultivate an
appreciation for the literature, and we will also learn about the
authors of these papers.

A successful student will:
\begin{enumerate}
    \item Gain an appreciation for mathematical literature.
    \item Learn to see the beauty of a well formed proof.
    \item Learn to write beautiful proofs yourself.
    \item Classify problems into decidable and undecidable problems.
    \item Understand how computational theory relates to mathematics,
        the universe, and human beings.
    \item Get a taste of the peer-review process.
    \item Understand the underlying theory of computation.
    \item View computer science as a mathematical discipline.
    \item Realize that mathematicians are not divine beings who hand
        wisdom down from on high.
    \item Learn to participate in making mathematical discoveries.
\end{enumerate}

The more ambitious among you may even try to produce publishable results. 

% Fifth Section %%%%%%%%%%%%%%%%%%%%%%%%%%%%%%%%%%%%%%%%%%%
\section*{Course Structure}
\subsection*{Methods of Instruction}
\begin{itemize}
    \item Lecture
    \item Reading
    \item Peer Review
    \item Discussion
    \item Presentations
\end{itemize}

\subsection*{Grading}
Let $A$ be a sequence $A=\langle \epsilon, \Sigma,
\alpha, \phi, \rho, \sigma \rangle$ where 

\begin{itemize}
    \item $\epsilon := $ Sequence of grades $\langle
        \epsilon_1, \epsilon_2, \dots, \epsilon_{|\epsilon|} \rangle$
        pertaining to grades on easy problems
    \item $\Sigma := $ Sequence of grades $\langle \Sigma_1,
        \Sigma_2, \dots, \Sigma_{|\Sigma|}\rangle$ pertaining to
        grades on harder problems
    \item $\alpha := $  $\displaystyle\frac{|\epsilon|
        + |\Sigma| + |X|}{|P|}$ where $X$ is a set of attempted but
        not solved problems and $P$ is the set of all problems
        presented to the class.
    \item $\phi := $ A sequence of grades $\langle \phi_1,
        \phi_2 \rangle$ pertaining to two peer reviewed papers
    \item $\rho := $ The set of all contributions made to the
        class, either through posing problems, peer review, or
        discussion.
    \item $\sigma := $ Is the grade of your final research
        presentation.
\end{itemize}

We define a function $g : A \mapsto x \in \mathbb{R} | 0 \leq
x \leq 1$.  Futhermore, we construct a Turing machine
$Z_g$ which computes the function $g$.

In order to arrive at the sought after function, $Z_g$
first computes $A'$ where $A'=\langle e, s, \alpha, p,
f, r, \sigma \rangle$ where each $x \in A'$
corresponds to the like term in $A$, and where every
$x \in A'$ satisfies the constraints $x \in
\mathbb{R}| 0 \leq x \leq 1$.  The simplest of these
transformations are $\alpha$ and $\sigma$, which
are simply a direct transcriptions from $A$.  The
transformations for $e,s,f$ are accomplished through simple
averaging of their corresponding sequences.  This leaves the
computation of $p$ and $r$.  Unfortunately,
these mappings require a qualitative analysis which is not readily
obtainable via conventional means due to the ambiguity of its
definition and must be approximated by a stochastic process (your
professor).  Of course, this establishes that your grades are not
strictly computable, and the proceeding can only therefore be an
approximation of $g$, but as it is the best we can do we
must proceed!  Having established that $Z_g$ cannot exist,
we replace it with $\hat{Z_g}$ which approximates
$g$ via the means described above and with the addition of
the approximate terms.

Having approximated $A'$, we introduce $W=\langle 10,
20, 10, 30, 10, 20\rangle$.  Because $W$ is
a sequence of constants, it may be encoded into the internal
configuration of $\hat{Z_g}$.  The final computation in
this approximation is performed by arithmetic and is well defined as:
\[
\displaystyle\frac{\sum_{i=1}^7 A'_i W_i}{100}
\]

The real number obtained by this computation will be converted into
one of the arbitrary elements in the grade alphabet using the
well-known method.

% Course Schedule %%%%%%%%%%%%%%%%%%%%%%%%%%%%%%%%%%%%%%%%%%%
\subsection*{Schedule}
This is the tentative schedule for our course.  Papers are introduced
on the days bearng their full title, and discussions follow.

\subsubsection*{January 2020}
\begin{tabular}{rrrrrrr}
Su & Mo & Tu & We & Th & Fr & Sa\\
   &    &    &  1 &  2 &  3 &  4\\ 
 5 &  6 &  7 &  8 &  9 & 10 & 11\\ 
12 & 13 & 14 & 15 & 16 & 17 & 18\\ 
19 & 20 & 21 & 22 & 23 & 24 & 25\\ 
26 & 27 & 28 & 29 & 30 & 31 &\\
\end{tabular}

\begin{itemize}
\item\textbf{Wed January  8} - Introduction \& Cantor
\item\textbf{Fri January 10} - {\em On an Elementary Question in the
Theory of Manifolds} by Georg Cantor
\item\textbf{Mon January 13} - Cantor Discussion
\item\textbf{Wed January 15} - Cantor Discussion
\item\textbf{Fri January 17} - On the Foundational Crisis and the Hilbert Program
\item\textbf{Mon January 20} - {\em Principia Mathematica} 
\item\textbf{Wed January 22} - Discussion on Principia
\item\textbf{Fri January 24} - Discusiion on Principia
\item\textbf{Mon January 27} - {\em On Formally Undecidable
Propositions of Principia Mathematica} by Kurt G\"odel
\item\textbf{Wed January 29} - G\"odel Discussion
\item\textbf{Fri January 31} - G\"odel Discussion
\end{itemize}
\hrulefill

\subsubsection*{February 2020}
\begin{tabular}{rrrrrrr}
Su & Mo & Tu & We & Th & Fr & Sa\\
   &    &    &    &    &    &  1\\ 
 2 &  3 &  4 &  5 &  6 &  7 &  8\\ 
 9 & 10 & 11 & 12 & 13 & 14 & 15\\ 
16 & 17 & 18 & 19 & 20 & 21 & 22\\ 
23 & 24 & 25 & 26 & 27 & 28 & 29\\ 
\end{tabular}
\begin{itemize}
\item\textbf{Mon February  3} - G\"odel Discussion
\item\textbf{Wed February  5} - G\"odel Discussion
\item\textbf{Fri February  7} - G\"odel Discussion
\item\textbf{Mon February 10} - {\em An Unsolvable Problem of
Elementary Number Theory} by Alonzo Church
\item\textbf{Wed February 12} - Church Discussion
\item\textbf{Fri February 14} - Church Discussion
\item\textbf{Mon February 17} - Church Discussion
\item\textbf{Wed February 19} - Church Discussion
\item\textbf{Fri February 21} - Church Discussion
\item\textbf{Mon February 24} - {\em On Compuable Numbers With an
Application to the Entscheidungsproblem} by Alan Turing
\item\textbf{Wed February 26} - Turing Discussion
\item\textbf{Fri February 28} - Turing Discussion
\end{itemize}
\hrulefill

\subsubsection*{March 2020}
\begin{tabular}{rrrrrrr}
Su & Mo & Tu & We & Th & Fr & Sa\\
 1 &  2 &  3 &  4 &  5 &  6 &  7\\ 
 8 &  9 & 10 & 11 & 12 & 13 & 14\\ 
15 & 16 & 17 & 18 & 19 & 20 & 21\\ 
22 & 23 & 24 & 25 & 26 & 27 & 28\\
29 & 30 & 31 &    &    &    & \\
\end{tabular}

\begin{itemize}
\item\textbf{Mon March  2} - Turing Discussion
\item\textbf{Wed March  4} - Turing Discussion
\item\textbf{Fri March  6} - Turing Discussion
\item\textbf{Mon March  9} - Exploring Computability and Proof
\item\textbf{Wed March 11} - Exploring Computability and Proof
\item\textbf{Fri March 13} - Exploring Computability and Proof
\item\textbf{Mon March 16} Spring Break
\item\textbf{Wed March 18} Spring Break
\item\textbf{Fri March 20} Spring Break
\item\textbf{Mon March 23} - {\em Three Models for the Description of
Langauge} by Noam Chomsky
\item\textbf{Wed March 25} - Chomsky Discussion
\item\textbf{Fri March 27} - {\em On Certain Formal Properties of Grammars}  by Noam Chomsky
\item\textbf{Mon March 30} - Chomsky Discussion
\end{itemize}
\hrulefill


\subsubsection*{April 2020}
\begin{tabular}{rrrrrrr}
Su & Mo & Tu & We & Th & Fr & Sa\\
   &    &    &  1 &  2 &  3 &  4\\
 5 &  6 &  7 &  8 &  9 & 10 & 11\\
12 & 13 & 14 & 15 & 16 & 17 & 18\\
19 & 20 & 21 & 22 & 23 & 24 & 25\\ 
26 & 27 & 28 & 29 & 30 &    &\\ 
\end{tabular}
\begin{itemize}
\item\textbf{Wed April  1} - Chomsky Discussion
\item\textbf{Fri April  3} - Chomsky Discussion
\item\textbf{Mon April  6} - {\em The Complexity of Theorem-Proving
Procedures} by Stephen Cook
\item\textbf{Wed April  8} - Cook Discussion
\item\textbf{Fri April 10} Good Friday - College Closed
\item\textbf{Mon April 13} - {\em Universal Sequential Search Problems
} by Leonid Levin
\item\textbf{Wed April 15} - The Cook Levin Theorem
\item\textbf{Fri April 17} - The Cook Levin Theorem
\item\textbf{Mon April 20} - The Cook Levin Theorem
\item\textbf{Wed April 22} - The Cook Levin Theorem
\end{itemize}
\hrulefill


% Fifth Section %%%%%%%%%%%%%%%%%%%%%%%%%%%%%%%%%%%%%%%%%%%

\newpage
\section*{Course Policies}

\subsection*{Late Policy}
No late work will be accepted under any circumstances (except as mercy
and decency may dictate in extremely rare events).

\subsection*{Extra Credit}
No extra credit will be given under any circumstances.

\subsection*{Excused Absences}
In some cases, absences may be excused. These include:
\begin{itemize}
    \item School Sanctioned Events (Sports, Concerts, etc.)
    \item Severe Illness
    \item Family Emergencies
    \item Court Appearance / Jury Duty
\end{itemize}
In the case of a school event, notice must be given at least one week
prior to the absence. The notice must include a signed note from the
faculty or staff member in charge of the event. This note must be
given in physical form, electronic notes will not be accepted.
In the case of illness, a doctor's note is required. Note that
except in extreme circumstances, doctor's appointments do not qualify as a valid reason to miss
a class. Please be respectful of the other students, and schedule
appointments during your free time.

Family emergencies will require some form of proof. Where possible,
you must give advance notice of missing a class. The exception to this
would need to be fairly severe, and hopefully it will not come up.
For court appearances and/or jury duty, you must provide a copy of
your summons. You may redact any details you wish, save for the
actual date and time of your appearance. Court appearances must be
cleared at least one week in advance.

\subsection*{Communication and Extra Help}
You are always welcome at office hours for help with any
questions you may have about the course. For help at other times during the day, stop by or call my office to see if I'm available. You can also contact me by email, but often I can better help you face to face and may respond with a request that
you come to see me. Note that I do not typically respond to email between 5 p.m. and 8 a.m. You may make appointments to see me at other times if your schedule does not permit you to attend my office hours.


\subsection*{Plagiarism and Cheating}
You are expected to do your own work. Never submit the work of others,
never give unauthorized assistance to others, do not use unauthorized
aids during exams, and do not ask for help from other
faculty members without the approval of your professor. Plagiarism and cheating are serious offenses that will not be
tolerated. Explanations regarding these offenses and how they are handled can be found in the MC Student Handbook at\newline
https://www.maryvillecollege.edu/academics/catalog/handbook/section-nine/.\newline
You are expected to read and understand these policies. Offenses on specific assignments, quizzes, or exams will result
in a score of 0 on the relevant assignment, and a letter of censure will be placed in your college file. Repeat offenses will
result in further disciplinary action, including the possibility of failing the course.

\subsection*{Students with Disabilities}
Any student who feels s/he may need learning or physical
accommodation(s) based on the impact of a disability should contact Services for Students with Disabilities to discuss your
specific needs. Please contact 981-8124 to coordinate reasonable accommodations for students with documented
disabilities. The Disability Services office is located in the Learning Center in the basement of Thaw Hall. Undocumented
disabilities will not be accommodated.

\end{document}
